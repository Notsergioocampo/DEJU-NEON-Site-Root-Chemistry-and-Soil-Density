% Options for packages loaded elsewhere
\PassOptionsToPackage{unicode}{hyperref}
\PassOptionsToPackage{hyphens}{url}
%
\documentclass[
]{article}
\usepackage{amsmath,amssymb}
\usepackage{iftex}
\ifPDFTeX
  \usepackage[T1]{fontenc}
  \usepackage[utf8]{inputenc}
  \usepackage{textcomp} % provide euro and other symbols
\else % if luatex or xetex
  \usepackage{unicode-math} % this also loads fontspec
  \defaultfontfeatures{Scale=MatchLowercase}
  \defaultfontfeatures[\rmfamily]{Ligatures=TeX,Scale=1}
\fi
\usepackage{lmodern}
\ifPDFTeX\else
  % xetex/luatex font selection
\fi
% Use upquote if available, for straight quotes in verbatim environments
\IfFileExists{upquote.sty}{\usepackage{upquote}}{}
\IfFileExists{microtype.sty}{% use microtype if available
  \usepackage[]{microtype}
  \UseMicrotypeSet[protrusion]{basicmath} % disable protrusion for tt fonts
}{}
\makeatletter
\@ifundefined{KOMAClassName}{% if non-KOMA class
  \IfFileExists{parskip.sty}{%
    \usepackage{parskip}
  }{% else
    \setlength{\parindent}{0pt}
    \setlength{\parskip}{6pt plus 2pt minus 1pt}}
}{% if KOMA class
  \KOMAoptions{parskip=half}}
\makeatother
\usepackage{xcolor}
\usepackage[margin=1in]{geometry}
\usepackage{graphicx}
\makeatletter
\def\maxwidth{\ifdim\Gin@nat@width>\linewidth\linewidth\else\Gin@nat@width\fi}
\def\maxheight{\ifdim\Gin@nat@height>\textheight\textheight\else\Gin@nat@height\fi}
\makeatother
% Scale images if necessary, so that they will not overflow the page
% margins by default, and it is still possible to overwrite the defaults
% using explicit options in \includegraphics[width, height, ...]{}
\setkeys{Gin}{width=\maxwidth,height=\maxheight,keepaspectratio}
% Set default figure placement to htbp
\makeatletter
\def\fps@figure{htbp}
\makeatother
\setlength{\emergencystretch}{3em} % prevent overfull lines
\providecommand{\tightlist}{%
  \setlength{\itemsep}{0pt}\setlength{\parskip}{0pt}}
\setcounter{secnumdepth}{-\maxdimen} % remove section numbering
% definitions for citeproc citations
\NewDocumentCommand\citeproctext{}{}
\NewDocumentCommand\citeproc{mm}{%
  \begingroup\def\citeproctext{#2}\cite{#1}\endgroup}
\makeatletter
 % allow citations to break across lines
 \let\@cite@ofmt\@firstofone
 % avoid brackets around text for \cite:
 \def\@biblabel#1{}
 \def\@cite#1#2{{#1\if@tempswa , #2\fi}}
\makeatother
\newlength{\cslhangindent}
\setlength{\cslhangindent}{1.5em}
\newlength{\csllabelwidth}
\setlength{\csllabelwidth}{3em}
\newenvironment{CSLReferences}[2] % #1 hanging-indent, #2 entry-spacing
 {\begin{list}{}{%
  \setlength{\itemindent}{0pt}
  \setlength{\leftmargin}{0pt}
  \setlength{\parsep}{0pt}
  % turn on hanging indent if param 1 is 1
  \ifodd #1
   \setlength{\leftmargin}{\cslhangindent}
   \setlength{\itemindent}{-1\cslhangindent}
  \fi
  % set entry spacing
  \setlength{\itemsep}{#2\baselineskip}}}
 {\end{list}}
\usepackage{calc}
\newcommand{\CSLBlock}[1]{\hfill\break\parbox[t]{\linewidth}{\strut\ignorespaces#1\strut}}
\newcommand{\CSLLeftMargin}[1]{\parbox[t]{\csllabelwidth}{\strut#1\strut}}
\newcommand{\CSLRightInline}[1]{\parbox[t]{\linewidth - \csllabelwidth}{\strut#1\strut}}
\newcommand{\CSLIndent}[1]{\hspace{\cslhangindent}#1}
\ifLuaTeX
  \usepackage{selnolig}  % disable illegal ligatures
\fi
\usepackage{bookmark}
\IfFileExists{xurl.sty}{\usepackage{xurl}}{} % add URL line breaks if available
\urlstyle{same}
\hypersetup{
  pdftitle={DEJU NEON Site Root Chemistry and Soil Density},
  pdfauthor={Sergio O.},
  hidelinks,
  pdfcreator={LaTeX via pandoc}}

\title{DEJU NEON Site Root Chemistry and Soil Density}
\author{Sergio O.}
\date{April 11, 2025}

\begin{document}
\maketitle

\section{Introduction}\label{introduction}

Comprehending the belowground nutrient cycling is pivotal for assessing
how ecosystems can store carbon, support plant growth, and respond to
environmental changes. In terrestrial systems such as those limited by
cold temperatures and resource scarcity, plant roots represent the
primary interface for vegetation and solid environments, thus mediating
water and nutrient uptake, further influencing microbial community
dynamics, and contributing to the long-term organic formations in the
soil. Arctic tundra soils are primarily made of decomposing plant
detritus. The microbial decomposition of detritus material plays a
central role in regulating carbon and nitrogen availability (Weintraub
and Schimel 2003). Since most microbial activity in these soils is
heterotrophic and thus dependent on plant-derived carbon, the amount and
the quality of root inputs, also known as organic matter, are central to
shaping the availability of biologically accessible nitrogen.
Interactions between root traits and microbial functions must form the
biochemical building blocks of carbon and nitrogen turnover in
ecosystems that identify as boreal. Understanding the elemental
composition of roots, specifically carbon and nitrogen content, thus not
only mirrors the physiological conditions of vegetation and provides
critical information for broader biogeochemical research towards
nutrient-scarce environments.

As stated, roots play an important role in water absorption and
facilitating nutrients; however, they also influence belowground
biological processes in essential ways for maintaining plant homeostasis
in a changing ecosystem. Plant roots in terrestrial systems link
vegetation and soil environments, entirely shaping the microbial
communities around them and contributing to organic matter inputs.
Mediating the biochemical exchanges supports plant growth (Zhalnina et
al. 2018). Reinforcing the idea that beyond their passive contribution
of organic plant matter, roots continue to alter the biological
properties and its nearest rhizosphere in order to create a local
condition that can regular the microbial activity and nutrient
availability in an area so scarce as mentioned before (Day et al. 2010).
Chemical compounds such as carbon and nitrogen make plant structure and
metabolism possible, shaping long-term soil fertility and biogeochemical
cycling.

Nitrogen is critical for synthesising various essential nutrients such
as amino acids, nucleic acids, and even enzymes, reinstating its primary
determining factor of metabolic functions and productivity of plants.
For many ecosystems, nitrogen availability caps growth while also in
excess can achieve the same conclusion, especially in cold and
nutrient-poor soil. Carbon provides the foundation for plant biomass and
is necessary for developing and stabilising soil containing organic
matter. Substantial portions of photosynthetically fixed carbon are
estimated to be around 40\% to 73\% allocated belowground, highlighting
the important role of roots in carbon cycling within ecosystems (Day et
al. 2010). Measurements of percent nitrogen and percent carbon in roots
can offer indispensable insight towards both the physiological state of
the plant and or the nutrient status of the soil in which it resides.

The carbon to nitrogen (C:N) ratio is an informative metric that
indicates the balance between energy-rich carbon compounds and
nutrient-abundant nitrogen. This ratio is commonly used in this field of
research to predict decomposition rates of organic matter, microbial
accessibility or organic substrates, and broader ecosystem nutrient
dynamics and fluctuations. Roots with a higher C:N ratio tend to
decompose slowly, demonstrating a slower turnover of carbon and nitrogen
pools within the soil. These effects depend on root size, usually finer
roots with lower C:N ratios that contribute significantly to nitrogen
cycling in forest and tundra systems (Träger et al. 2017). Recent
ecosystem modelling has depicted fine root traits being influential
towards predicting soil carbon storage and heterotrophic respiration,
particularly within functional plant types within tundra ecosystems such
as sedges and grasses (Euskirchen et al. 2022). Boreal systems have been
observed to have a trend where carbon derived from roots decomposes
substantially slower than lead litter, contributing to longer-term
nitrogen retention within the soil organic matter (Kyaschenko et al.
2019). Some variables that may influence root elemental composition are
temperature, moisture availability, and soil texture. Nitrogen addition
reduces fine root biomass and surface area towards surface-level soil
layers, significantly altering nutrient availability below ground root
traits (Yan et al. 2017). Variations towards the soil properties such as
moisture, pH, ammonium concentration, and even carbon-to-nitrogen ratios
have been shown to influence the composition and distribution of soil
microbial communities, indicating belowground nutrient availability and
biogeochemistry are indeed linked to biological structure (Shi et al.
2015). Soil bulk density is a special integrative variable analysed in
this research, defined as the mass of dry soil per unit volume,
porosity, and mineral composition. High bulk density typically signifies
compacted soil with low pore space, correlation with restricted root
growth, limited aeration, and reduced microbial activity. Soil
compaction results mainly from anthropogenic disturbances such as urban
development, constraining root growth and limiting physical expansion
and access to essential resources (Day et al. 2010). In contrast, low
bulk-density soils tend to be more biologically active and structurally
favourable for many root developments and microbial colonisations. Bulk
density can be an intermediate for physical constraints towards nutrient
uptake and root decay processes.

Using the standardised data from the National Ecological Observatory
Network (NEON), a nationwide infrastructure program designed explicitly
for ecological monitoring, investigations towards these relationships
were analysed. This study focused on NEON's Delta Junction Site (DEJU),
located in the lower east part of Alaska within the boreal forest, a
tundra transition zone of domain 19 (Taiga). The DEJU site is identified
by its subarctic climate, discontinuous permafrost, and nutrient-limited
soils. These provide a unique setting to evaluate how plant roots
allocate and store carbon and nitrogen under extreme environmental
conditions. Research objectives address the following questions: How do
root nitrogen, carbon and C:N ratios vary across root size classes such
as ≤4mm against \textgreater4mm? And do these nutrient concentrations
show meaningful association towards soil bulk density, used here as an
intermediate for physical soil constraints

\section{Methods}\label{methods}

All data for this study were accessed and processed using R, a free,
open-source statistical programming environment. neonUtilities packages
were used to efficiently download, stack, and organise data from the
National Ecological Observatory Network (NEON). The analysis primarily
focused on two NEON data products collected from soil megapit protocols
at the Delta Junction (DEJU) site in Alaska.

The first dataset, DP1.10066.001 Root Biomass and Chemistry, includes
the biomass estimate and elemental composition data towards root
fragments extracted from different soil horizons within a megapit
sample. Primary variables of interest were percent carbon, percent
nitrogen, and the derived carbon-to-nitrogen ratios. The second dataset
was DP1.00096.001, the soil's physical and Chemical Properties,
providing information on the bulk density, soil horizon depth, and
corrections for the coarse fragments, a necessary component for
estimating true soil compaction and porosity.

Data was retrieved using the loadByProduct() function, which tells the
specific site - ``DEJU'' and check.size = FALSE to automatically process
all available data from the DEJU site. This function returns a named
list of data frames containing stacked tables for every data product,
consolidating observations across the years and sampling events into a
singular format.

To classify root samples via size, we extracted the mpr\_carbonNitrogen
table from the root chemistry product and joined it with
mpr\_perrootsample metadata table using the shared sampleID identifier.
sizeCategory variable designated root fragments as either ``fine'' if it
was (≤4 mm or ``coarse'' if they were \textgreater4 mm. Using this data,
they were categorised for comparison for nutrient concentrations. Soil
dataset using mgp\_perbulksample table was used to pull bulk density
values. Values were adjusted for the volume of coarse fragments when
needed, following NEON's post-processing protocols. Based on the depth
range or the horizon ID, the bulk density values were joined to the root
chemistry data, depending on the metadata availability and structure.

The data was cleaned before analysis. To prevent statistical distortion,
the merged dataset was filtered to exclude samples with missing values,
including values with a zero for \%C, \%N, or bulk density. The data
analysis was explored to assess the distribution of values and identify
potential outliers. Summary statistics were calculated for each root
size category, including group means and standard errors.

Visualisations were produced using ggplot2 packages. Histograms of
nitrogen content and C:N ratios, scatter plots comparing \%C and \%N,
soil bulk density versus nutrient content, and grouped bar charts
displaying mean values by root size with error bars indicating one
standard error of the mean is made by ggplot2 packages. All the code
used for data processing, analysis, and rendering for visualisation was
written in R Markdown to ensure full reproducibility and transparency of
the research workflow.

\subsection{Site Locations and
Descriptions}\label{site-locations-and-descriptions}

This study used data from the Delta Junction (DEJU) field site, one of
the core terrestrial locations monitored by NEON. DEJU, to be precise,
is approximately 150 kilometres southeast of Fairbanks and falls within
the Taiga NEON Domain 19. This domain represents the northern boreal
forest transition zone. The tundra is characterised by a subarctic
climate with long cold winters, short growing seasons, and extremely
scarce nutrient resources.

DEJU site occupies a boreal coniferous biome where the vegetation is
dominated by the black spruce (Picea mariana) and white spruce (Picea
glauca) along with low shrubs, sedges, mosses, and dwarf birch (Betula
glandulosa). These well-adapted species can handle the region's low
temperatures, waterlogged soils, and frequent nutrient bottlenecks. The
site is underlain by discontinuous permafrost, creating highly variable
vertical patterns in soil temperature, moisture, and microbial activity.

Soils at DEJU are classified as gelisols because their cold climate
soils are characterised by perennially frozen sublayers, cryoturbation,
and organic-rich surface horizons. Edaphic properties within the soil
create strong physical and biochemical constraints for plant roots.
Usually, these soils are poorly drained, slow to thaw in spring, and low
in biologically disposable nitrogen and phosphorus. As a result, plant
root systems must operate under critical environmental stress. The
chemical composition of roots, such as their nitrogen and carbon
content, reflects adaptive strategies and biogeochemical responses under
these extreme conditions.

NEON technicians obtained all root and soil data used in this analysis
from a singular soil megapit excited near the site's eddy covariance
flux tower. The eddy covariance flux tower consistently measures carbon,
water, and energy exchanges. The megapit was sampled using NEON's
standardised soil horizon protocols. This ensured consistency across
sites and the representativeness of local vertical variations within the
soil properties.

DEJU lies in a high latitude, cold climate ecosystem with persistent
soil compaction, moisture saturation, and nutrient limitations,
providing the perfect setting for studying root nutrient dynamics under
environmental constraints. As climate proceeds to change, accelerating
permafrost thaw and altering structures of boreal ecosystems, baseline
measurements from the DEJU site offer invaluable insight towards root
systems shifting to soil conditions and nutrient regimes within the
northern biomes.

\subsection{Field Sampling Design}\label{field-sampling-design}

Field sampling was conducted by official NEON technicians using
standardised protocols to ensure consistency in methodology across all
NEON sites and sampling time points (Ayres 2019). At the Delta Junction
field sites, one soil megapit was excavated following NEON's soil
megapit characterisation protocol to obtain vertical profiles of root
biomass, chemical composition, and the soil's physical properties
(Keller et al. 2008). The megapit was located adjacent to DEJU's
instrumented eddy covariance flux tower. This allowed the sampling to
represent the same ecological footprint as the other ongoing carbon and
energy flux measurements (National Ecological Observatory Network 2017).

The pit was manually excavated using stratified sampling, with samples
collected at multiple depth intervals corresponding to soil horizons
such as organic, mineral, and transitional layers. Each soil horizon was
carefully delineated based on field observations of texture, colour, and
structure to allow for accurate vertical profiling of below-ground
traits.

Roots were manually extracted from the exposed soil wall using a
combination of sieving and hand-picking methods, following NEON's
megapit\_perrootsample and mpr\_carbonNitrogen protocols. The soil
matrix was then separated, and roots were gently cleaned with DI water
to remove remaining soil particles. They were stored under cold
conditions before processing.

Each root sample was categorised using the diameter and turned into a
standardised size class. This study grouped roots into two main
categories: fine roots for ≤ 4 mm and coarse roots for (\textgreater{} 4
mm. This is consistent with fields available for the sizeCategory
variable. At the same time, NEON sampling protocols could distinguish
between intermediate-size classes. Only fine and coarse classifications
were deployed in the analysis due to sample availability and
consistency.

Each sample was uniquely identified with a sampleID that encoded
metadata about horizon, depth interval, and size class. These
identifiers were important for linking data to their corresponding
chemical properties and joining data sets with soil physical data in
post-processing.

Within the laboratory, root samples were oven-dried and grounded before
elemental analysis. Percent carbon and nitrogen were quantified from dry
combustion, typically by a Carbon, Hydrogen, and Nitrogen analyser.
Carbon to nitrogen ratios were then derived and calculated from
mass-based values unless molar conversion was indicated in NEON's
records for that sampling year and that sire.

\subsubsection{Soil Sampling}\label{soil-sampling}

Undisturbed soil cores were collected to determine bulk density at
matching depth intervals near the zone. Samples were collected via
cylindrical sleeves inserted horizontally into the soil wall to minimise
compaction or volume distortion. Depth intervals were aligned with root
collection zones to allow accurate soil and root data pairing by
horizon.

Bulk density was calculated following NEON protocol using the formula,
Bulk Density = Oven Dry mass of soil (volume of core - volume of coarse
fragments). This correction ensured gravel, rocks, or root segments did
not artificially reduce bulk density estimates. All soil samples were
dried at 105 degrees Celsius until mass stabilisation was achieved,
after which coarse fragments of \textgreater2mm were sieved and
subtracted from the total core volume.

All field and laboratory data were entered into NEON's centralised
database system. Standardised sampleID tags enabled precise
cross-referencing between root chemical traits and soil physical
properties, supporting integrative analyses of how soil structure
influences root elemental composition within the vertical soil profile.

\subsection{Data Analysis and
Visualization}\label{data-analysis-and-visualization}

I used the ggplot2 package (Wickham 2016) in R to do my visualizations.

\section{Results}\label{results}

The root nitrogen concentration at the DEJU site demonstrated moderate
variability across samples, with most values ranging from approximately
0.3\% to 1\%. Distributions of nitrogen content were notably
right-skewed (Figure 1), exhibiting the majority of root samples being
relatively nitrogen-poor, which is consistent with expectations for
nutrient-limited boreal soils.

A percent comparison between carbon and nitrogen revealed a weak
positive linear relationship (Figure 2). However, the trendline
suggested that roots with higher carbon content also tended to have
slightly more elevated nitrogen levels. Scatter plots were substantial,
making the correlation weak. This could suggest that while some
co-regulation of carbon and nitrogen storage could occur within roots,
the data suggest it is more plausible mediated by other biological or
environmental factors.

The Calculated carbon-to-nitrogen ratio depicted extreme variation, with
most values falling between 40 and 80 (Figure 3). These radically high
and variable ratios strongly indicate slow decomposition potential and
reduced microbial availability, which aligns with expectations for cold,
carbon-abundant root tissues in boreal ecosystems.

When ranked by root size category, fine roots ≤ 4 mm displayed a
slightly greater average nitrogen content and lower C:N ratio than
coarse roots \textgreater{} 4 mm (Figure 4). Percent carbon was
relatively consistent across size classes, indicating that structural
investments in carbon are not correlated by diameter. At the same time,
nitrogen content may be more related to metabolic activity, which is
usual within finer roots.

No relationships of interest were observed between soil bulk density or
any of the measured root traits, including nitrogen seen in (Figure 5),
Carbon (Figure 6), and C:N ratios (Figure 7). Linear trend lines for all
comparisons were almost flat, meaning there was little to no association
between soil compaction and root elemental composition for the same
profile sample.

\section{Discussion}\label{discussion}

This study offers insights into belowground nutrient dynamics at the
Delta Junction (DEJU) NEON site, especially concerning the elemental
composition of root tissue and its potential association with the soil's
physical conditions. Nitrogen and carbon content variations were
observed across the root size classes between coarse and fine. There
were no strong relationships between the bulk density and root chemical
composition. These findings indicate that within the conditions sampled,
bulk density is not a primary driver of \%C, \%N, or C:N ratios in plant
roots at this site.

One explanation for the lack of associations could be the limited
variation in bulk density within sampled megapits. Due to discontinuous
permafrost, DEJU soils are cold, high in organic matter, and frequently
exhibit cryoturbation. Conditions such as these could result in
relatively homogenous soil structure across horizons, reducing the
potential to detect fine-scale gradients in bulk density. Corrections
for coarse fragments and inconsistencies in depth alignment between root
and soil data may have introduced additional variability that masked
subtle differences.

Other environmental drivers, such as soil temperature, moisture
availability, microbial biomass, and species-level root traits, have
more substantial implications towards nutrient allocation than bulk
density alone. Plant-microbe interactions, root age, and tissue
longevity are commonly identified in shaping nutrient level
concentrations within roots, especially in low nitrogen systems,
particularly in boreal forests. Microbial community composition has been
indicated to have variance systematically along environmental gradients
even without visible plant community differences, pointing out the
critical but often unseen role of microbial structures towards driving
ecosystem-level nutrient dynamics (Zimmerman and Vitousek 2012). NEON
dataset used in this study, however, did not include microbial or
taxonomic information for root samples, which limits attributed
abilities for variations to these important biotic factors

Observations that fine roots contain slightly more nitrogen and exhibit
lower C:N ratios than coarse roots are consistent with well-established
functional distinctions in the root systems. Fine roots are highly
metabolically active, have shorter lifespans, and are primarily
responsible for nutrient acquisition. Meanwhile, coarse roots serve
structural and transport roles with lower nitrogen investments. A modest
positive correlation between \%C and \%N across all root samples
supports that these elements may co-accumulate in actively growing
tissues. The strength of the relationship, however, was weak and not
robust in a statistical manner.

High variability and right-skewed nitrogen concentration distribution
and C:N ratios reinforce substantial heterogeneity in root physiology
condition, turnover, and decomposition status. Variation could reflect
differences in species composition, developmental stages, or localised
soil conditions not captured in the current metadata. NEON does not
provide taxonomy resolution for root fragments collected from megapits,
so it is not possible to assess interspecific variation, which is an
important factor in trait-based and community-level ecological analyses
Critical limiting factors for this study are all root and soil samples
originating from a singular megapit, limiting spatial replication and
broader generalizability of results. Although the DEJU site is
ecologically representative of the boreal forest permafrost transition
zone, discoveries presented here should not be calculated for NEON sites
in other domains or biomes without precaution. Additionally, relatively
small sample sizes, particularly for nitrogen content, bottlenecks in
statistical power, and reduced confidence in identifying subtle
ecological patterns.

Regardless of these constraints, the study provides indispensable
baseline insight towards root nutrient composition in a high-altitude,
nutrient-deprived ecosystem. These findings confirm that intrinsic plan
factors such as root size and function are more strongly associated with
elemental composition than bulk soil physical structures. That being
said, it has to be considered within at least the range of bulk density
values encountered at this site. These results emphasise root trait
variation and the importance of understanding nutrient cycling within
cold climate ecosystems. They also point out the need for more spatially
replicated, species-resolved studies to enhance inferencing at broader
ecological scales.

\section{Acknowledgment}\label{acknowledgment}

This study was made possible using chatGPT for the code, and publicly
available ecological data provided by the National Ecological
Observatory Network (NEON), a program sponsored by the National Science
Foundation and operated under cooperative agreement by Battelle Memorial
Institute. I thank Dr.~Zimmerman, the University of San Francisco
Department of Biology, and any teaching assistants for their support and
guidance throughout this project. I also acknowledge the contributions
of NEON field staff at the Delta Junction (DEJU) site for data
collection and curation.

\section{Figures}\label{figures}

\includegraphics{Final_Report_files/figure-latex/Nitrogen-Distrubution-1.pdf}

\textbf{Figure 1:} Histogram demonstrates the distribution of nitrogen
content \%N in the root samples collected via the Delta Junction (DEJU)
site. Values range from approximately 0.2\% to 1.4\%, with a
right-skewed distribution indicating generally low nitrogen
concentrations in roots.

\includegraphics{Final_Report_files/figure-latex/Carbon-vs-Nitrogen-1.pdf}

\textbf{Figure 2:} Scatter plot of root percent carbon \%C versus
percent nitrogen \%N for all sampled root fragments. A weak positive
linear trend is visible, however high variability suggests only a
somewhat modest association between carbon and nitrogen concentrations.

\includegraphics{Final_Report_files/figure-latex/C-N-Ratio-1.pdf}

\textbf{Figure 3:} Histogram of calculated carbon-to-nitrogen C:N ratios
across all root samples. Values vary drastically, with most between 25
and 90, reflecting substantial heterogeneity in root nutrient allocation
and decomposition potential.

\includegraphics{Final_Report_files/figure-latex/Megapit-root-carbon-nitrogen-consumption-as-Nitrogen-vs-Carbon-1.pdf}

\textbf{Figure 4:} Bar plot comparing mean percent carbon, \%C blue, and
percent nitrogen, \%N red, across two root size categories ≤4 mm and
\textgreater4 mm at the DEJU site. Coarse roots which are \textgreater4
mm, exhibit a higher average carbon concentration at 49.1\% when
compared to fine roots exhibiting 40\%. In contrast, fine roots ≤4 mm
display slightly higher nitrogen content at 0.7\% relative to coarse
roots displaying at 0.6\%. Error bars represent ±1 standard error of the
mean.

\includegraphics{Final_Report_files/figure-latex/Soil-Bulk-Density-vs-Root-Nitrogen-1.pdf}

\textbf{Figure 5:} Scatter plot showing the relationship between soil
bulk density and root nitrogen content in percent. Root nitrogen
concentrations range from approximately 0.25\% to 1.75\%, with no clear
trend across the bulk density gradient. The flat linear regression line
indicates no meaningful association between soil compaction and root
nitrogen levels within the sampled data.

\includegraphics{Final_Report_files/figure-latex/Soil-Bulk-Density-vs-Root-Carbon-1.pdf}

\textbf{Figure 6:} Scatter plot shows the relationship between soil bulk
density and root carbon content in percent. Root carbon concentrations
range from approximately 5\% to 55\%, but no discernible pattern is
observed across the bulk density gradient. The flat regression line and
broad vertical scatter plots indicate no meaningful association between
soil compaction and carbon concentration in root tissues within the
sampled data.

\includegraphics{Final_Report_files/figure-latex/Soil-Bulk-Density-vs-Root-C-N-Ratio-1.pdf}

\textbf{Figure 7:} Scatter plot demonstrates the relationship between
soil bulk density and root carbon-to-nitrogen C:N ratio. Root C:N ratios
range widely, with values typically between 25 and 100 but few outliers
around 200 in a few cases. The regression line is flat, indicating no
meaningful association between soil bulk density and root C:N ratio in
the sampled data.

\section{New Ecologically Useful
Visualizations}\label{new-ecologically-useful-visualizations}

\includegraphics{Final_Report_files/figure-latex/Root-CN-vs-Depth-1.pdf}

\textbf{Figure 8:} Scatter plot showing the relationship between root
C:N ratio and soil depth. Points are colored by depth to highlight
vertical patterns in root nutrient allocation.

\includegraphics{Final_Report_files/figure-latex/Boxplots-SizeCategory-Nitrogen-1.pdf}

\textbf{Figure 9a:} Boxplot comparing nitrogen content between fine
(≤4mm) and coarse (\textgreater4mm) root size categories. Points show
individual samples with jitter to avoid overplotting.

\includegraphics{Final_Report_files/figure-latex/Boxplots-SizeCategory-CN-1.pdf}

\textbf{Figure 9b:} Boxplot comparing C:N ratios between fine (≤4mm) and
coarse (\textgreater4mm) root size categories. Points show individual
samples with jitter to avoid overplotting.

\textbf{Figures 9a and 9b:} Boxplots comparing nitrogen content (9a) and
C:N ratios (9b) between fine (≤4mm) and coarse (\textgreater4mm) root
size categories. Points show individual samples with jitter to avoid
overplotting. Fine roots typically show higher nitrogen content and
lower C:N ratios compared to coarse roots.

\includegraphics{Final_Report_files/figure-latex/BulkDensity-Distribution-1.pdf}

\textbf{Figure 10:} Histogram showing the distribution of soil bulk
density values across all samples from the DEJU site.

\includegraphics{Final_Report_files/figure-latex/Carbon-Nitrogen-by-Size-1.pdf}

\textbf{Figure 11:} Scatter plot of carbon versus nitrogen content,
colored by root size category. Fine roots (≤4mm) are shown in blue,
coarse roots (\textgreater4mm) in orange, with separate regression lines
for each size class.

\includegraphics{Final_Report_files/figure-latex/CN-vs-BulkDensity-by-Size-1.pdf}

\textbf{Figure 12:} Scatter plot examining the relationship between root
C:N ratio and soil bulk density, with points colored by root size
category. Fine roots (≤4mm) are shown in blue, coarse roots
(\textgreater4mm) in orange, with separate regression lines for each
size class.

\textbf{Figure 13:} Multi-panel depth profile showing mean values of
(top) root C:N ratio, (middle) root nitrogen content, and (bottom) soil
bulk density across depth intervals. Depth increases downward following
ecological convention.

\section*{Sources Cited}\label{sources-cited}
\addcontentsline{toc}{section}{Sources Cited}

\phantomsection\label{refs}
\begin{CSLReferences}{1}{0}
\bibitem[\citeproctext]{ref-ayres2019quantitative}
Ayres, E. 2019. Quantitative guidelines for establishing and operating
soil archives. Soil Science Society of America Journal 83:973--981.

\bibitem[\citeproctext]{ref-day2010tree}
Day, S. D., P. E. Wiseman, S. B. Dickinson, and J. R. Harris. 2010. Tree
root ecology in the urban environment and implications for a sustainable
rhizosphere. Arboriculture \& Urban Forestry (AUF) 36:193--205.

\bibitem[\citeproctext]{ref-euskirchen2022assessing}
Euskirchen, E. S., S. P. Serbin, T. B. Carman, J. M. Fraterrigo, H.
Genet, C. M. Iversen, V. Salmon, and A. D. McGuire. 2022. Assessing
dynamic vegetation model parameter uncertainty across alaskan arctic
tundra plant communities. Ecological Applications 32:e2499.

\bibitem[\citeproctext]{ref-keller2008continental}
Keller, M., D. S. Schimel, W. W. Hargrove, and F. M. Hoffman. 2008. A
continental strategy for the national ecological observatory network.
The Ecological Society of America: 282-284.

\bibitem[\citeproctext]{ref-kyaschenko2019soil}
Kyaschenko, J., O. Ovaskainen, A. Ekblad, A. Hagenbo, E. Karltun, K. E.
Clemmensen, and B. D. Lindahl. 2019. Soil fertility in boreal forest
relates to root-driven nitrogen retention and carbon sequestration in
the mor layer. New Phytologist 221:1492--1502.

\bibitem[\citeproctext]{ref-neon2017}
National Ecological Observatory Network. 2017. Megapit soil
characterization protocol. \url{https://data.neonscience.org/documents}.

\bibitem[\citeproctext]{ref-shi2015vegetation}
Shi, Y., X. Xiang, C. Shen, H. Chu, J. D. Neufeld, V. K. Walker, and P.
Grogan. 2015. Vegetation-associated impacts on arctic tundra bacterial
and microeukaryotic communities. Applied and Environmental Microbiology
81:492--501.

\bibitem[\citeproctext]{ref-trager2017potential}
Träger, S., A. Milbau, and S. D. Wilson. 2017. Potential contributions
of root decomposition to the nitrogen cycle in arctic forest and tundra.
Ecology and Evolution 7:11021--11032.

\bibitem[\citeproctext]{ref-weintraub2003interactions}
Weintraub, M. N., and J. P. Schimel. 2003. Interactions between carbon
and nitrogen mineralization and soil organic matter chemistry in arctic
tundra soils. Ecosystems 6:0129--0143.

\bibitem[\citeproctext]{ref-ggplot2}
Wickham, H. 2016. \href{https://ggplot2.tidyverse.org}{ggplot2: Elegant
graphics for data analysis}. Springer-Verlag New York.

\bibitem[\citeproctext]{ref-yan2017spatial}
Yan, G., F. Chen, X. Zhang, J. Wang, S. Han, Y. Xing, and Q. Wang. 2017.
Spatial and temporal effects of nitrogen addition on root morphology and
growth in a boreal forest. Geoderma 303:178--187.

\bibitem[\citeproctext]{ref-zhalnina2018dynamic}
Zhalnina, K., K. B. Louie, Z. Hao, N. Mansoori, U. N. Da Rocha, S. Shi,
H. Cho, U. Karaoz, D. Loqué, B. P. Bowen, and others. 2018. Dynamic root
exudate chemistry and microbial substrate preferences drive patterns in
rhizosphere microbial community assembly. Nature microbiology
3:470--480.

\bibitem[\citeproctext]{ref-zimmerman2012endophytes}
Zimmerman, N. B., and P. M. Vitousek. 2012.
\href{https://doi.org/10.1073/pnas.1209872109}{Fungal endophyte
communities reflect environmental structuring across a hawaiian
landscape}. Proceedings of the National Academy of Sciences
109:13022--13027.

\end{CSLReferences}

\end{document}
